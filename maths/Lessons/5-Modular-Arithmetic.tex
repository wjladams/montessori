\documentclass[11pt]{article}
\usepackage{fullpage}
\usepackage{graphicx}
\newcommand{\mblank}{\hbox{\underline{\hspace{0.5in}}}}
\title{Intoduction to Modular Arithmetic, mod 10}
\author{Exercises}
\begin{document}
\maketitle
\section{Introduction}
Modular arithmetic is a very useful way of working with numbers.  It is
based on looking at the remainder of a number when divided by another
number.  Here are some examples:
\begin{enumerate}
	\item 25 divided by 10 has a remainder of 5, we write as
	$$25 \equiv 5 (\hbox{mod } 10)$$
	\item 20 divided by 10 has a remainder of 0 we can write as:
	$$20 \equiv 0 (\hbox{mod } 10)$$
	\item 36 divided by 10 has a remainder of 6, we can write as:
	$$36 \equiv 6 (\hbox{mod } 10)$$
\end{enumerate}

\section{Exercises}
Fill in the blanks for each of the following:
\begin{enumerate}
	\item $23 \equiv \mblank{}(\hbox{mod }10)$
	\item $57 \equiv \mblank{}(\hbox{mod }10)$
	\item $124 \equiv \mblank{}(\hbox{mod }10))$
	\item $31+15 \equiv \mblank{}(\hbox{mod }10))$
	\item $\mblank{} \equiv 7 (\hbox{mod } 10)$
	\item $\mblank{} \equiv 3 (\hbox{mod } 10)$
	\item $9+8 \equiv \mblank{}(\hbox{mod } 10)$
	\item $9+1 \equiv \mblank{}(\hbox{mod } 10)$
	\item $6+\mblank{} \equiv 0(\hbox{mod } 10)$
	\item $3+\mblank{} \equiv 0(\hbox{mod } 10)$
\end{enumerate}


\end{document}
